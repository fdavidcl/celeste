\documentclass[11pt]{article}
\usepackage{dsfont}
\usepackage{hyperref}
\usepackage{dsfont}
\usepackage{mathtools}
\usepackage[utf8]{inputenc}
\usepackage[spanish]{babel}
\usepackage{enumerate}
\usepackage{graphicx}
\usepackage{amsfonts}
\usepackage{cite}
\usepackage{amsthm}
\newtheorem{lema}{Lema}[section]
\newtheorem{prop}{Proposición}[section]
\theoremstyle{definition}
\newtheorem{mydef}{Definición}[section]
\newtheorem{ejemplo}{Ejemplo}[section]

\title{Las isometrías en la Mecánica Celeste}
\author{Minia Bermúdez de la Puente García \\ Francisco David Charte Luque \\ José Ángel Garrido Calvo}
\date{7 de octubre de 2016}


\begin{document}
\maketitle

\vphantom{A}

\tableofcontents
\newpage

\section{Repaso de las isometrías de $\mathbb{R}^3$}
	\subsection{Isometrías lineales}

    \noindent\textbf{Notación.}\quad Dada una aplicación lineal $T$ notaremos indistintamente $T$ tanto a la aplicación como a su matriz asociada en la base usual de $\mathbb R^3$.
    \\

    Entre dos espacios vectoriales euclídeos abstractos se definen las isometrías como aquellas biyecciones que conservan la distancia, y las isometrías lineales son los isomorfismos que preservan el producto escalar. En estos apuntes realizamos una definición más concreta para el caso de $\mathbb R^3$.

    \begin{mydef}
    Una isometría de $\mathbb R^3$ es una aplicación lineal $A:\mathbb R^3\rightarrow \mathbb R^3$, $A(x)=Ax\ \forall x\in \mathbb{R}^3$, verificando $\left<Ax,Ay\right>=\left<x,y\right>\forall x,y\in \mathbb R^3$. Equivalentemente, $A$ conserva módulos, es decir, $|Ax|=|x|\ \forall x\in\mathbb R^3$.
    \end{mydef}

    Como consecuencia directa de la definición tenemos que el determinante de la matriz asociada a una isometría es 1 o -1. Llamaremos a la isometría \textbf{directa} o \textbf{inversa} respectivamente.

	\begin{prop}
    Una aplicación lineal $A:\mathbb R^3\rightarrow\mathbb{R}^3$ es una isometría lineal $\Leftrightarrow$ su matriz asociada es ortogonal (es decir, $AA^t=I$).
    \end{prop}
	\begin{prop}
    El conjunto de todas las aplicaciones lineales de un espacio vectorial euclídeo $V$, $O(V)$, tiene estructura de grupo con la composición.
    \end{prop}

    % referencia: http://www.uv.es/ivorra/Libros/Geometria2.pdf

	A continuación se clasifican las isometrías de $\mathbb R^3$ y se dan algunos ejemplos.

    \subsection{Clasificación de isometrías lineales}
Sea $A:\mathbb R^3\rightarrow\mathbb R^3$ una isometría lineal. Consideremos el subespacio de vectores fijos que genera:
\[V_A=\left\{x\in\mathbb R^3/Ax=x \right\},\]
Observamos que este subespacio se corresponde con el conjunto de vectores propios asociados al valor propio 1 de A. Para estudiar su dimensión, la podemos obtener de la forma: $\mathrm{dim}(V_A)=3-\text{rango}(A-I)$. Asi, tendremos los siguientes casos.

    \subsubsection{Identidad}
    Si $\mathrm{dim}(V_A)=3$, entonces $A$ deja fijo todo $\mathbb R^3$. Por tanto, se trata de la identidad,  $A=I:\mathbb R^3\rightarrow \mathbb R^3, I(x)=x\ \forall x\in \mathbb R^3$.

    \subsubsection{Simetría respecto de un plano}
  	Si $\mathrm{dim}(V_A)=2$, $A$ tiene el plano $V_A$ como subespacio de vectores fijos. En este caso, si $B=\{v_1, v_2\}$ es una base ortonormal de dicho plano y $v_3$ es un vector unitario ortogonal al plano, tenemos la base de $\mathbb{R}^3$ $B'=\{v_1,v_2,v_3\}$ de forma que
    \[M(A, B')=\begin{pmatrix}1&0&0\\0&1&0\\0&0&-1\end{pmatrix}.\]

    Por tanto, $A$ lleva cada vector de $\mathbb R^3$ en su simétrico por el plano $V_A$.

\begin{ejemplo}
Sea $B=\{e1 , e2 , e3\}$ la base usual de  $\mathbb{R}^3$. La simetría respecto al plano generado por los ejes X e Y tiene por matriz asociada:
\[M(S_{X,Y},B)=
    \begin{pmatrix}
    \ 1 & 0 & 0 \\ 0 & 1 & 0 \\ 0 & 0 & -1 \\
    \end{pmatrix}.
  \]
\end{ejemplo}

\subsubsection{Rotación respecto de una recta}
En el caso en que $\mathrm{dim}(V_A)=1$, el subespacio $V_A$ es una recta vectorial. Sea $v$ un vector unitario director de la recta, y sean $u_1,u_2$ dos vectores unitarios ortogonales entre sí y a $v$, consideramos la base $B=\{u_1,u_2,v\}$ de $\mathbb R^3$ en la que la matriz asociada a $A$ se expresa:
    \[M(A, B)=\begin{pmatrix}a&b&0\\c&d&0\\0&0&1\end{pmatrix}.\]
Por la condición de ortogonalidad de la matriz, es fácil comprobar que tendrá una expresión del tipo
    \[\begin{pmatrix}\cos(\alpha)&-\sen(\alpha)&0\\\sen(\alpha)&\cos(\alpha)&0\\0&0&1\end{pmatrix}\]
    para conveniente $\alpha\in \mathbb R\setminus\{0\}$. Se trata, por tanto, de la rotación de ángulo $\alpha$ alrededor del eje de giro dado por la recta $V_A$.


\begin{ejemplo}
Sea $B=\{e1 , e2 , e3\}$ la base usual de  $\mathbb{R}^3$. La rotación de $\frac\pi 2$ respecto del eje Z tiene por matriz asociada:
\[M(R_{\pi/2},B)=
    \begin{pmatrix}
    \ 0 & -1 & 0 \\ 1 & 0 & 0 \\ 0 & 0 & 1 \\
    \end{pmatrix}.
  \]
\end{ejemplo}

\noindent\textbf{Observación.}\quad Las simetrías respecto de una recta son un caso particular de rotaciones para $\alpha=\pi$.
    \\

\subsubsection{Composición de una rotación y una simetría}

Si $\mathrm{dim}(V_A)=0$, entonces $A$ tiene al $0$ como único vector fijo. es una composición de una rotación y una simetría con eje de giro y plano de simetría perpendiculares entre sí. Si $u_1, u_2$ forman una base ortonormal del plano y $v$ es un vector unitario director del eje, la isometría tendrá la siguiente matriz en la base $B=\{u_1,u_2,v\}$:

    \[M(A, B)=\begin{pmatrix}\cos(\alpha)&-\sen(\alpha)&0\\\sen(\alpha)&\cos(\alpha)&0\\0&0&-1\end{pmatrix},\]
donde $\alpha$ es el ángulo de giro.

\begin{ejemplo}
Sea $B=\{e1 , e2 , e3\}$ la base usual de  $\mathbb{R}^3$. La simetría respecto del origen se corresponde con la rotación de $\pi$ alrededor del eje Z seguido por la simetría respecto del plano formado por los ejes X, Y:
\[M(S_0,B)=
    \begin{pmatrix}
    \ 1 & 0 & 0 \\ 0 & 1 & 0 \\ 0 & 0 & -1 \\
    \end{pmatrix}
    \begin{pmatrix}
    \ -1 & 0 & 0 \\ 0 & -1 & 0 \\ 0 & 0 & 1 \\
    \end{pmatrix}=
    \begin{pmatrix}
    \ -1 & 0 & 0 \\ 0 & -1 & 0 \\ 0 & 0 & -1 \\
    \end{pmatrix}.
  \]
\end{ejemplo}

\noindent\textbf{Observación.}\quad En general, las simetrías respecto de un punto en $\mathbb R^3$ son casos particulares de esta situación.
    \\

\subsection{Isometrías afines (Movimientos rígidos)}
	\begin{mydef}
    Un movimiento rígido es una aplicación $\phi:\mathbb R^3\rightarrow\mathbb R^3$ que conserva las distancias.
    \end{mydef}
    \begin{lema}
    Todo movimiento rígido $\phi$ viene determinado por una isometría lineal $A:\mathbb R^3\rightarrow\mathbb R^3$ y un vector $b\in \mathbb R^3$ de la siguiente forma: $\phi(x)=Ax+b\ \forall x\in\mathbb R^3$. Llamamos a $A$ \textbf{isometría (lineal) asociada a $\phi$}.
    \end{lema}
  \subsubsection{Isometría}
  La identidad, las rotaciones y las simetrías son movimientos rígidos trivialmente.
  \subsubsection{Traslación}
  Las traslaciones son los movimientos rígidos tales que la isometría asociada es la identidad.
  \subsubsection{Movimiento helicoidal}
  Un movimiento helicoidal es una composición de una rotación y una traslación.
  \subsubsection{Simetría deslizante}
  Una simetría deslizante es una composición de una simetría y una traslación.
  \subsubsection{Composición de rotación y simetría}
  Las composiciones de rotación (o movimiento helicoidal) y simetría (deslizante o no) son un movimiento rígido,
  cuya isometría asociada es la composición de rotación y simetría lineales asociadas al eje de giro y al plano de simetría.

  \begin{prop}
  Cualquier movimiento rígido es de uno de los tipos mencionados.
  \end{prop}

  % referencia: http://www.ugr.es/%7Ecrosales/1516/cys/tema0.pdf

\section{Uso en la teoría de Mecánica Celeste}

    \subsection {Demostración de un pequeño lema}
    \begin{lema} Sea $A \in \mathbb{M}_{3} \left( \mathbb{R} \right)$ una isometría y  $x\in \mathbb{R}^3$ una solución del problema de fuerzas centrales \begin{equation}\ddot{x}=f(|x(t)|)\frac{x(t)}{|x(t)|}, \label{pfc}\end{equation} entonces Ax también es solución.
    \begin{proof}
    Para que $Ax$ sea solución, debe cumplir la ecuación dada en \ref{pfc}:

    \begin{align*}
	\frac{d^2}{dt^2}\left( Ax \right)=A\ddot{x}=Af(|x(t)|)\frac{x(t)}{|x(t)|}\stackrel{(*)}{=}f(|Ax(t)|)\frac{Ax(t)}{|Ax(t)|} \quad \forall t\in \mathbb{R}^{+},
    \end{align*}
    donde en $(*)$ se usa que la isometría preserva los módulos.

    \end{proof}
\end{lema}

\section{Referencias}

\begin{itemize}
\item\textbf{Geometría} - Carlos Ivorra - \url{http://www.uv.es/ivorra/Libros/Geometria2.pdf}
\item\textbf{Repaso de aspectos de geometría afín euclídea} - César Rosales - \url{http://www.ugr.es/~crosales/1516/cys/tema0.pdf}
\end{itemize}


\end{document}
